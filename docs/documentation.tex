\documentclass{article}

% ############### %
% PACKAGE IMPORTS %
% ############### %
\usepackage[utf8]{inputenc}
%\usepackage[margin=80pt]{geometry}
\usepackage{fancyhdr}
\usepackage{graphicx}
\usepackage[hidelinks]{hyperref}
\usepackage[most]{tcolorbox}
\usepackage{tikz}
\usetikzlibrary{shadows}

\pagestyle{fancy}

% ################ %
% ENVIRONMENT DEFS %
% ################ %
\newenvironment{guide}[1]
{
	\begin{center}
		\begin{tcolorbox}[%
			colback=black!20, 
			boxrule=0pt, 
			title=Step-by-step: #1,
			enhanced,
			breakable,
			overlay unbroken={%
                \draw[line width=1pt, black, rounded corners]
        	    (frame.north west) rectangle (frame.south east);
			},
    		overlay first={%
        		 \draw[line width=1pt, black, rounded corners]
        	    (frame.south west) -- (frame.north west) -- (frame.north east) -- (frame.south east);
                \draw[line width=1pt, black]
                (frame.south west) -- (frame.south east);
            },
    		overlay middle={%
                \draw[line width=1pt, black]
        	    (frame.north west) rectangle (frame.south east);
        	},
    		overlay last={%
                \draw[line width=1pt, black, rounded corners]
        	    (frame.north west) -- (frame.south west) -- (frame.south east) -- (frame.north east);
                \draw[line width=1pt, black]
                (frame.north west) -- (frame.north east);
           	}
        ]{}
    	\begin{enumerate}
}
{
    		\end{enumerate}
    	\end{tcolorbox}
	\end{center}  	 
}

\newcommand{\guideimage}[1]
{
	\begin{center}
		\includegraphics[width=0.5\textwidth]{#1}
	\end{center}
}


% Adapted from this StackExchange post:
% https://tex.stackexchange.com/a/5227
\newcommand*\keystroke[1]
{
	\raisebox{-1.5pt}
	{
		\hspace{-8pt}
		\begin{tikzpicture}
		\node
		[
			draw,
			fill=black!65,
			drop shadow={shadow xshift=0.25ex,shadow yshift=-0.25ex,fill=black,opacity=0.75},
      		rectangle,
      		rounded corners=2pt,
      		inner sep=3pt,
      		outer sep=0pt,
      		line width=0.5pt,
      		font=\scriptsize\sffamily,
      		text=black!10
    	]
    	{
    		\hspace*{0.5pt}\tt #1\hspace*{0.5pt}
    	}
    	;
    	\end{tikzpicture}
		\hspace{-8pt}
  	}
}

% ############## %
% VARIABLE SETUP %
% ############## %
\title{Blender Tools for GFS Documentation}
\author{Pherakki}
\date{v0.1}


% ############## %
% DOCUMENT START %
% ############## %
\begin{document}
\maketitle
\pagenumbering{gobble}
\clearpage

\tableofcontents
\clearpage

\pagenumbering{arabic} 

\section{Getting Started}
BlenderToolsForGFS is a \textbf{Blender 2.81+} plugin. It should work on all versions of Blender including and beyond 2.81. The plugin has been developed on versions 2.83 and 3.4.1.

\subsection{Installing the Plugin}


\subsection{Removing the Plugin}

\subsection{A Note on Expectations}
This plugin should be viewed as a \textbf{supplement} to existing model-editing tools, and not as a replacement. You should, in particular, post-process any models exported from Blender in \textbf{GFD Studio}, following existing tutorials and knowledge to achieve your goals.

Please also note that although this document does offer a number of guides on bits of basic Blender usage required to use the plugin, this is not a guide on how to use Blender. If you are intending to work with models and animations, it is ultimately your responsibility to learn how to use Blender using the wealth of resources available on the internet.

This is also not a guide on how to mod games. Again, it is your responsibility to seek out or discover the information you require to attain any such goals.

What this guide \textit{is} intended for is to assist you in successfully exporting data from Blender to the GFS file format. Suggestions for improvements on how to do that are very much welcomed, as long as they do not fall out-of-scope for the essential use of the plugin.
\clearpage

\section{Import}
\subsection{Import Restrictions}
Currently, a subset of features of the GFS format are not supported by the plugin and cannot be exported. These are:
\begin{itemize}
\item File versions outside of the range 0x01104920 - 0x01105100
\item EPL Data (Subfiles / Particle Effects)
\item Morphs / Shapekeys
\item Strings that are not SHIFT-JIS- (for texture file names) or UTF8- (everything else) encoded
\end{itemize}

There are also some additional considerations to bear in mind.
\begin{itemize}
\item During import, meshes are re-parented from their positioning node to the root node, whilst preserving their world transform. This is because there does not seem to be a straightforward way to make a mesh follow the transform of a bone whilst also being deformed by an armature without Blender double-counting transformations. These positioning nodes are also removed from the import if they are not used for any other purpose.
\item Animations of the Root Node and of mesh positioning nodes are not editable from Blender. Any mesh positioning node animations \textbf{will be messed up} by the Blender import if these nodes were not originally children of the root node. This must be fixed outside of Blender by re-parenting and repositioning these nodes.
\item Only non-BC7 DDS texture slots are currently importable in models.
\end{itemize}

\clearpage 

\subsection{Importing Model (GMD/GFS) Files}
\begin{guide}{Accessing the Model Import Menu}
\item Open the File menu and navigate to the \textbf{Import} \textgreater\space \textbf{GFS} \textgreater\space \textbf{GFS Model} submenu item.
\guideimage{images/import/import_gmd.png}
\item Select the GMD or GFS file to import using the file browser that pops up.
\end{guide}

\clearpage

\subsection{Importing Animation (GAP) Files}
\begin{guide}{Accessing the Animation Import Menu}
\item Ensure that you have first imported the model that the animation is modelled for.
\item Open the File menu and navigate to the \textbf{Import} \textgreater\space \textbf{GFS} \textgreater\space \textbf{GAP Animation} submenu item.
\guideimage{images/import/import_gap.png}
\item Select the armature for your imported model using the drop-down.
\guideimage{images/import/import_gap_armature_select.png}
\item Select the GAP file to import using the file browser that pops up.
\end{guide}

\clearpage

\subsection{Blender Settings}
\subsubsection{Previewing Materials}
Many beginners to Blender are confused by the fact that models do not display their materials when it is first opened. Blender by default renders models in a ``Solid" representation that is useful for modellers editing meshes. You can preview materials instead by changing this render setting.
\begin{guide}{Previewing Materials}
\item Locate the \textbf{Viewport Shading} widget and select \textbf{Material Preview} mode.
\guideimage{images/import/import_preview_materials.png}
\end{guide}

\subsubsection{Importing Large Models}
\label{SECTION::ImportingLargeModels}
Some models, such as Field Models, are typically in such large units that they exceed the default render distance of Blender. In this instance, you may find it useful to increase Blender's render distance.\\
\begin{guide}{Increasing Blender's Render Distance}
\item Press the \keystroke{N} key to open the \textbf{Sidebar}.
\guideimage{images/import/import_field_open_menu.png}
\item Click the \textbf{View} tab in the \textbf{Sidebar}.
\guideimage{images/import/import_field_open_view.png}
\item Change the value in the \textbf{End} box to a value large enough for you to comfortably navigate the model.
\guideimage{images/import/import_field_set_distance.png}
\item Press the \keystroke{N} key to close the \textbf{Sidebar}.
\end{guide}

\subsubsection{Hiding Unused Bones}
On import, three bone layers will be created in the armature:
\begin{itemize}
\item All Bones
\item Bones used in Vertex Groups
\item Bones not used in Vertex Groups
\end{itemize}
Selecting one of these layers will allow you to hide these subsets of bones. Note that these groups are created \textbf{by the plugin} on import, and if you add new bones to a model then it is up to you to add bones to whatever bone layers you want.

\begin{guide}{Selecting Bone Layers}
\item Switch to Object Mode.
\guideimage{images/import/import_to_object_mode.png}
\item Select the model armature.
\item Click the Armature icon in the Properties Panel. Select the Bone Layer that is active in the Viewport.
\guideimage{images/import/import_bone_layer_select.png}
\end{guide}

\clearpage

\section{Editing Models and Animations}
\subsection{GFS Models}
Models are imported as Armature objects with meshes, cameras, and lights parented below them. Most of the objects imported from GFS files can carry additional information beyond pure geometry and shading data, which is outlined in the proceeding sections.

GFS models are very large compared to typical Blender scales. If parts of the model disappear as you zoom out, you will need to change the render distance of Blender as described in section \ref{SECTION::ImportingLargeModels}.

The data in the files are assumed to be oriented Y-up and with X-axis-oriented bones. During import, these are converted to the Z-up orientation and Y-axis-oriented bones to match the Blender conventions. On export, the Blender data is converted back to Y-up orientation and X-axis-oriented bones. 

\subsubsection{Bones}
\label{SECTION::EditingBones} 
There are no special considerations for bones beyond how they behave in Blender. However, bones do also have some additional information that can be attached to them not representable in Blender. These can be accessed from the \textbf{GFS Node} panel in the \textbf{Bone Properties}.

\begin{itemize}
\item \textbf{Unknown Float}: Unknown. Always seems to take a value of 1.
\item \textbf{Properties}: Custom properties attached to the bone. See section \ref{SECTION::GfsProperties} for further details.
\end{itemize}

\begin{guide}{Accessing Extra Bone Properties}
\item Select the armature of the model either in the Viewport or in the Outliner.
\guideimage{images/editing_models/edits_select_armature.png}
\item Switch to Pose Mode.
\guideimage{images/editing_models/edits_to_pose_mode.png}
\item Select a bone and select the Bone icon in the Properties panel. You will find a panel called ``GFS Node" containing the additional Bone Properties.
\guideimage{images/editing_models/edits_bone_properties.png}
\end{guide}
\clearpage

\subsubsection{Rest Pose}

\subsubsection{Meshes}
Meshes that belong to a GFS model are parented under an armature. Meshes can use an armature deform with vertex groups as usual inside Blender. However, each vertex is only allowed to be part of a maximum of 4 vertex groups, and a maximum of 256 vertex groups are permitted across the entire model. If two meshes are weighted to the same bone, this counts as two vertex groups, since vertex groups are tied to the transform of a mesh in the GFS file format. The two vertex groups can be counted as one if meshes are parented to each other, as will be described momentarily.

The transform of the mesh inside Blender is exported as the mesh transform within the GFS files. Note however that meshes can be parented under other meshes to ensure that they share a transform. In this situation, the transform of the child-mesh is ignored and should always be a unit transform if you want to accurately see what will be exported in the viewport. In addition, if a mesh and its parent-mesh both use the same bone as a vertex group, it will be counted once instead of double-counted since both meshes have the same transform. Only one level of mesh-to-mesh parenting will be detected on export.

\begin{guide}{Mesh Parenting for GFS}
\item Meshes must be parented under an armature. The transform of a mesh dictates the transform of the mesh in the exported file.
\guideimage{images/editing_models/edits_parent_mesh.png}
\item Meshes can also be parented under other meshes. The transform of these child meshes will be ignored. Parenting a mesh under another mesh ensures that they will share a transform in the exported file and vertex groups used by both meshes will not be double-counted by the file format.
\guideimage{images/editing_models/edits_child_mesh.png}
\end{guide}

\begin{guide}{Unparenting Objects within Blender}
\item Select an object in the outliner or in the viewport.
\item Press \keystroke{Alt} + \keystroke{P} with the mouse in the Viewport.
\item Select either:
\begin{enumerate}
\item \textbf{Clear Parent} if you want to unparent the object from its parent \textbf{and} move the mesh so that its transform is relative to the origin.
\item \textbf{Clear and Keep Transform} if you want to unparent the object from its parent \textbf{and} edit the mesh's transform so that it stays in the same place in the viewport.
\end{enumerate}
Ignore \textbf{Clear Parent Inverse}.
\guideimage{images/editing_models/editing_unparent.png}
\end{guide}

\begin{guide}{Parenting Objects within Blender}
\item Select the child object in the outliner or in the viewport.
\guideimage{images/editing_models/edits_parent_1.png}
\item Select the parent object in the outliner or in the viewport with \keystroke{Ctrl} + Click.
\guideimage{images/editing_models/edits_parent_2.png}
\item Press \keystroke{Ctrl} + \keystroke{P} with the mouse in the Viewport.
\item Select either:
\begin{enumerate}
\item \textbf{Object (Without Inverse)} if you want to parent the object and reset its transform.
\item \textbf{Object (Keep Transform Without Inverse)} if you want to parent the object \textbf{and} edit the mesh's transform so that it stays in the same place in the viewport.
\end{enumerate}
\guideimage{images/editing_models/edits_parent_3.png}
\item In any situation, if after parenting your child object's transform is not what you expected, then:
\begin{itemize}
\item Select the child object.
\item Press \keystroke{Alt} + \keystroke{P} with the mouse in the Viewport.
\item Click \textbf{Clear Parent Inverse}.
\end{itemize}
This will remove the hidden ``parent inverse" matrix that sometimes gets inserted when parenting objects. This is harmless, but merely means that you object's transform may not align with what you see in the viewport if it is not a unit transform.
\guideimage{images/editing_models/editing_unparent.png}
\end{guide}

Meshes have certain attributes that may be exported. These are:
\begin{itemize}
\item \textbf{Export Bounding Box}/\textbf{Sphere}: Define a bounding Box / Sphere on export.
\item \textbf{Export Normals}/\textbf{Tangents}/\textbf{Binormals}: Export these attributes. They are required for specific material options to work.
\item \textbf{Unknown Flag}: The purpose of Unknown Flags is not known. Checking and unchecking these may cause or fix crashes.
\item \textbf{Unknown 0x12}: Purpose unknown.
\item The GFS Node sub-panel will appear if the Mesh is not parented under another Mesh. Refer to section \ref{SECTION::EditingBones} for further details.
\end{itemize}

\begin{guide}{Accessing Extra Mesh Attributes}
\item Select the mesh in the Outliner or in the Viewport.
\guideimage{images/editing_models/edits_select_mesh.png}
\item Switch to Object Mode.
\guideimage{images/editing_models/edits_to_object_mode.png}
\item Select the Mesh icon in the Properties panel. You will find a panel called ``GFS Mesh" containing the additional Mesh attributes.
\guideimage{images/editing_models/edits_mesh_properties.png}
\end{guide}

\underline{\textbf{WARNINGS}}
\begin{itemize}
\item Many tutorials and help articles will tell you to \textbf{Apply Transforms} to your mesh to reset their transforms to a unit transform. \textbf{This is dangerous}. When you \textbf{Apply Transforms}, you are translating, rotating, and scaling the vertices that make up the mesh, rather than applying an extra transform on top of the mesh. This means that, for example, your mesh will now use the origin as the reference point from which it rotates and scales, rather than the point around which the mesh was modelled.
\end{itemize}

\subsubsection{Materials}
\label{SECTION::EditingMaterials}

Materials are not yet sufficiently understood to a degree whereby they can be consistently rendered in Blender. Therefore, materials are only represented in Blender as the Diffuse Texture of the material and all other properties are inferred from attributes or the names of Shader Nodes. Edits to the GFS Material attributes will, more often than not, simply lead to the model crashing any game that loads it. Due to this, editing materials from Blender is \textbf{not} currently recommended, except for the purposes of:
\begin{enumerate}
\item Adding textures to a material.
\item Perhaps minor tweaks to attributes.
\item Researching what the attributes do.
\end{enumerate}
You will have much more success following the strategy of current model editing guides: post-processing the model in \textbf{GFD Studio} by copying materials from other models onto your exported model. How to do this is beyond the scope of this documentation but detailed tutorials are available online.

In the future, when materials are properly understood, a custom Shader Node Group should be implemented that allows the material to be rendered in a meaningful fashion, and allowing the values of attributes to be auto-calculated such that they do not crash the game. \textbf{There is not enough knowledge currently to do this and any contributions to material research enabling this feature is highly welcomed.}

There are then two essential pieces to Material editing:
\begin{itemize}
\item Texture Samplers
\item GFS Material Attributes
\end{itemize}

\paragraph{Texture Samplers:}You can edit the textures used by a material by accessing the Shader Nodes. The first step is to open the Shader Node Editor.
\begin{guide}{Opening the Shader Node Editor}
\item Select the mesh in the Outliner or in the Viewport.
\guideimage{images/editing_models/edits_select_mesh.png}
\item Switch to Object Mode.
\guideimage{images/editing_models/edits_to_object_mode.png}
\item Select the Material icon in the Properties panel.
\guideimage{images/editing_models/edits_select_material.png}
\item Select the Shader Editor.
\guideimage{images/editing_models/edits_open_shader_nodes.png}
\end{guide}

You can then set create new samplers to be exported and set their GFS attributes. There are nine types of textures available in the GFS format. Each material can have \textbf{one} of each type, and the plugin will recognise which type of sampler a node represents by the \textbf{name} of the node. The recognised names are:
\begin{itemize}
\item Diffuse Texture
\item Normal Texture
\item Specular Texture
\item Reflection Texture
\item Highlight Texture
\item Glow Texture
\item Detail Texture
\item Night Texture
\item Shadow Texture
\end{itemize}

Note also that the UV maps must also have specific names due to the way they are stored in the GFS file format. The permitted names are:
\begin{itemize}
\item UV0
\item UV1
\item UV2
\item UV3
\item UV4
\item UV5
\item UV6
\item UV7
\end{itemize}
meaning that up to 8 maps are allowed per mesh.

\begin{guide}{Editing Texture Samplers}
\item Inside the Shader Editor, select or create an Image Node.
\item Select the \textbf{Node} tab in the Sidebar. If the Sidebar is not open, press \keystroke{N} with the mouse inside the Shader Editor.
\item Set the name of the node to one of the nine accepted names by changing the value in the Name field.
\guideimage{images/editing_models/edits_rename_shader_node.png}
\item Select or create a UV Map node. Select a UV map present on the mesh from the drop-down on the node and hook the UV connector up to the Vector connector on the Texture node.
\guideimage{images/editing_models/edits_set_node_uv_map.png}
\item You can access the properties of a texture sampler by selecting the \textbf{GFS Texture} tab in the Sidebar. The properties for the sampler are given in the \textbf{GFS Texture Sampler} panel.
\guideimage{images/editing_models/edits_texture_sampler_properties.png}
\end{guide}

\paragraph{GFS Material Attributes:}
\noindent Materials have certain attributes that may be exported. These are:
\begin{itemize}
\item \textbf{Unknown Flags}: The purpose of Unknown Flags is not known. Checking and unchecking these may cause or fix crashes.
\item \textbf{Enable Vertex Colors}: Use Color Map data on the mesh.
\item \textbf{Enable UV Anims}: Allow the UV coordinates of the mesh to be animated.
\item \textbf{Use Light 2}: Purpose unknown.
\item \textbf{Purple Wireframe}: Render the mesh as a purple wireframe.
\item The GFS Node sub-panel will appear if the Mesh is not parented under another Mesh. Refer to section \ref{SECTION::EditingBones} for further details.
\end{itemize}
\begin{guide}{Accessing Extra Material Attributes}
\item Select the mesh in the Outliner or in the Viewport.
\guideimage{images/editing_models/edits_select_mesh.png}
\item Switch to Object Mode.
\guideimage{images/editing_models/edits_to_object_mode.png}
\item Select the Material icon in the Properties panel. You will find a panel called ``GFS Material" containing the additional Material attributes.
\guideimage{images/editing_models/edits_material_properties.png}
\end{guide}

\subsubsection{Textures}
Textures must be DDS images with either no compression, or DXT1, DX3, or DXT5 compression. BC7 textures cannot currently be loaded by Blender and are currently unsupported. In the future, BC7 textures should be loadable \textit{via} an automatic conversion to a Blender-readable format.

\noindent Materials have certain attributes that may be exported. These are:
\begin{itemize}
\item \textbf{Unknown 1}: Purpose unknown.
\item \textbf{Unknown 2}: Purpose unknown.
\item \textbf{Unknown 3}: Purpose unknown.
\item \textbf{Unknown 4}: Purpose unknown.
\end{itemize}
These attributes can be edited from the \textbf{GFS Texture} Sidebar panel on a Texture Image node in the Shader Editor, as described in section \ref{SECTION::EditingMaterials}.

\subsubsection{Cameras}
Cameras are constrained to a bone in the model armature with a \textbf{Child Of} constraint. Due to the way that bones are imported, \textbf{you will need a 90 degree rotation in the Z axis} on the camera to make it point ``horizontally". The camera will be exported in whatever orientation it is in within Blender, however \textbf{it is strongly recommended} to just keep a 90-degree Z-rotation on the camera and do all other positioning using the bone it is constrained to.

For a camera to be recognised for export, you must do one of the following:
\begin{itemize}
\item Constrain the camera to follow a bone with a ``Child Of" constraint.
\item Parent the Camera to a bone. Note that parenting a Camera to a bone will place the camera at the tail of the bone rather than the head, although this should not affect the pose in which the camera is exported.
\end{itemize}

Some properties of Blender cameras will be exported to GFS camera properties, and some properties are not represented in Blender. This is described below.
\begin{guide}{Accessing Camera Properties}
\item In the \textbf{Camera Data} Properties panel, there are a number of camera properties. In the \textbf{Lens} section, the \textbf{Type} must be set to \textbf{Perspective} and the \textbf{Lens Unit} to \textbf{Field of View}. The \textbf{Field of View} attribute will be exported, although it may not exactly match up with the in-game field-of-view in the current version of the plugin. Additionally, the \textbf{Clip Start} and \textbf{Clip End} fields will be exported and will appear in-game as they do in Blender.
\item Furthermore, there are two properties not represented in Blender that can be exported, found in the \textbf{GFS Camera}. \textbf{Aspect Ratio}s can be represented in Blender, but only for the entire scene and not for individual cameras, so this is not currently represented. \textbf{Unknown 0x50} is unknown.
\guideimage{images/editing_models/edits_camera_properties.png}
\end{guide}

\subsubsection{Lights}
Lights are constrained to a bone in the model armature with a \textbf{Child Of} constraint. Due to the way that bones are imported, \textbf{you will need a 90 degree rotation in the Z axis} on the camera to make it point ``horizontally". The light will be \textbf{exported assuming a 90-degree Z-rotation on the light}---do all other positioning using the bone it is constrained to.

For a light to be recognised for export, you must do one of the following:
\begin{itemize}
\item Constrain the Light to follow a bone with a ``Child Of" constraint.
\item Parent the Light to a bone. Note that parenting a Light to a bone will place the light at the tail of the bone rather than the head, \textbf{and the light will be repositioned to the head of the bone during export due to GFS file format constraints}.
\end{itemize}

Some properties of Blender lights will be exported to GFS light properties, and some properties are not represented in Blender. This is described below.
\begin{guide}{Accessing Light Properties}
\item In the \textbf{Light Data} Properties panel, there are a number of light properties.
The only Blender property currently exported is the \textbf{Color} property, which sets the RGB light color.
\item The additional properties are in the \textbf{GFS Light} panel. There are first three elements common to all lights:
\begin{itemize}
\item \textbf{Color Alpha} is the alpha channel of the light color.
\item \textbf{Color 1} is a color variable that does not seem to be used.
\item \textbf{Color 3} is a color variable that does not seem to be used.
\end{itemize}
There are then three kinds of light, selectable from the drop-down:
\begin{itemize}
\item \textbf{Type 1}: Unknown light type, does not appear to render in-game. Has no specific properties.
\item \textbf{Sphere}: A point light that can have an attentuation radius. This has the unknown properties \textbf{unknown 0x34}, \textbf{unknown 0x38}, and \textbf{unknown 0x3C}, plus an \textbf{unknown setting}. If the unknown setting is active, the point light will illuminate everything within the \textbf{inner radius}, and attenuate to zero illumination between the \textbf{inner radius} and \textbf{outer radius}. Switching the setting off instead provides the settings \textbf{unknown 0x48}, \textbf{unknown 0x4C}, and \textbf{unknown 0x50}. Attempts to get a light to render using these variables have been unsuccessful.
\item \textbf{Hemisphere}: A spot light that can have an attentuation radius. This has the unknown properties \textbf{unknown 0x54}, \textbf{unknown 0x58}, \textbf{unknown 0x5C}, \textbf{unknown 0x60}, \textbf{unknown 0x64}, \textbf{unknown 0x68}, \textbf{unknown 0x6C}, and \textbf{unknown 0x70}. This also has an \textbf{Unknown Setting} similar to the Sphere light. In principle these settings could be mapped to the Blender ``Spot Shape", but lights should be better understood overall before implementing this.
\end{itemize}
At the end of the properties is a list of flags. Only \textbf{flag 0} seems to be used amongst the provided flags, and its purpose is unknown.
\guideimage{images/editing_models/edits_light_properties.png}
\end{guide}


\subsubsection{Physics}
Model physics are not currently editable from Blender, but should be preserved for re-export.

\clearpage
\subsection{GFS Animations}
In this section, the three major kinds of animation are covered. Sections~\ref{SECTION::Edits::Animations::NormalAnimations}, \ref{SECTION::Edits::Animations::BlendAnimations}, and \ref{SECTION::Edits::Animations::LookAtAnimations} cover the properties and Blender data required for each of these types to be exported and displayed correctly. Section~\ref{SECTION::Edits::Animations::AnimationPacks} covers how to organise the set of animations in an Animation Pack that can be exported. In section~\ref{SECTION::Edits::Animations::TipsForEditing}, some useful information on editing Blend animations from within Blender is provided.

Currently, only \textbf{bone animations} are editable from within Blender. There are (at least) four other kinds of animation, but they are not exposed to the user for the following reasons:
\begin{itemize}
\item \textbf{Material Animations}: Not enough known about materials to create an animation node with animatable inputs.
\item \textbf{Camera Animations}: No strategy currently developed to link Camera object animations to the corresponding Armature animation.
\item \textbf{Morph Animations}: No strategy currently developed to link Shape Key animations to the corresponding Armature animation.
\item \textbf{Animation Type 5}: Unknown animation type, so nothing can be done with it currently. 
\end{itemize}

\subsubsection{Normal Animations}
\label{SECTION::Edits::Animations::NormalAnimations}
``Normal Animations" are regular animations. The animated quantities in these animations should override those that are present in the Rest Pose. In-game, all interpolations are done with \textbf{linear interpolation (LERP)}, except for quaternion animations which are done with \textbf{spherical linear interpolation (SLERP)}. For this reason, to ensure that the animation you see in Blender is accurately reflected in-game, you should ensure the following settings are applied:
\begin{itemize}
\item The Strip Blending should be set to \textbf{REPLACE}.
\item The keyframe interpolation should be set to \textbf{LINEAR} for all animations except for \textbf{rotation\_quaterion}, which should be set to \textbf{BEZIER}.
\end{itemize}
See Section~\ref{SECTION::Editing::Animations::TipsForEditing::AutoCorrect}, Subitem ``Auto-Correcting Actions" for information on how to apply these settings automatically.

\textbf{Normal Animations} have a set of properties that also need to be exported. Details on these are given below.

\clearpage

\begin{guide}{Accessing Normal Animation Properties}
\item The animation properties can be found by selecting a strip in the NLA editor and navigating to the ``GFS Animation" panel in the ``Strip" sidebar panel. You can use the \keystroke{N} key to open and close the sidebar whilst the mouse cursor is inside the NLA editor.
\item The \textbf{Category} should be set to ``Normal". Most of the \textbf{Unknown Flags} are unused. The first five flags are usually ticked if the following animations are present:
\begin{enumerate}
\item Flag 0 - Bone animations
\item Flag 1 - Material animations
\item Flag 2 - Camera animations
\item Flag 3 - Morph animations
\item Flag 4 - Unknown animations
\end{enumerate}
but since these do not seem to influence how animations are displayed and do not have an exact correspondence to these types of animation being present, they are left as unnamed flags for now.
\guideimage{images/editing_models/edits_normal_animations_1.png}
\item Further down, you can set ``Look At" animations for a Normal Animation. A \textbf{LookAt animation} is a kind of \textbf{Blend Animation} that makes a character look up, down, left, and right. Further details can be found in sections~\ref{SECTION::Edits::Animations::BlendAnimations} and \ref{SECTION::Edits::Animations::LookAtAnimations}. If you tick the box saying that an animation has lookat animations, you \textbf{must select all four animations you want to use as lookats} in order to successfully export. You can do this by clicking the animation boxes and selecting the animation that pops up. The only available animations are those that have been given the \textbf{LookAt} category.
The role of the LookAt factors is unknown.
\guideimage{images/editing_models/edits_normal_animations_2.png}
\end{guide}

\subsubsection{Blend Animations}
\label{SECTION::Edits::Animations::BlendAnimations}
\textbf{Blend Animations} are animations that are overlaid on \textbf{Normal Animations}. The animations are combined by combining each transformation channel individually. Positions and scales are added, and quaternions are combined with quaternion multiplication. Therefore, the Strip Blending type should be set to \textbf{COMBINE}. Keyframe interpolation should be handlded the same as the \textbf{Normal Animations}. However, Blender will attempt to \textbf{multiply} scale channels instead of adding them under the \textbf{COMBINE} Strip Blending mode. For this reason, scales should be separated to their own action with the \textbf{ADD} Strip Blending mode, and given the special \textbf{Blend Scale} category rather than the \textbf{Blend} category. Details on this are given in the Blend Properties Step-By-Step below, which also discusses other Blend animation properties. This means that any scales present in a \textbf{Blend Animation} will be ignored on export, and instead taken from a linked \textbf{Blend Scale Animation} if one is present.
\begin{guide}{Accessing Blend Animation Properties}
\item Test
\guideimage{images/editing_models/edits_blend_animations_1.png}
\item Test
\guideimage{images/editing_models/edits_blend_animations_2.png}
\end{guide}

\subsubsection{LookAt Animations}
\label{SECTION::Edits::Animations::LookAtAnimations}

\begin{guide}{Accessing LookAt Animation Properties}
\item Test
\guideimage{images/editing_models/edits_lookat_animations.png}
\end{guide}

\subsubsection{Animation Packs}
\label{SECTION::Edits::Animations::AnimationPacks}

\begin{guide}{Accessing Animation Pack Properties}
\item Test
\guideimage{images/editing_models/edits_animation_pack.png}
\end{guide}

\subsubsection{Tips for Editing Animations in Blender}
\label{SECTION::Edits::Animations::TipsForEditing}

\paragraph{Auto-Correcting Actions}
\label{SECTION::Editing::Animations::TipsForEditing::AutoCorrect}
\begin{guide}{Auto-Correcting Actions}
\item Test
\end{guide}

\paragraph{Previewing with the NLA Editor}
\label{SECTION::Editing::Animations::TipsForEditing::PreviewingNLA}
\begin{guide}{Previewing with the NLA Editor}
\item Test
\end{guide}

\paragraph{Previewing Blend Animations during Editing}
\label{SECTION::Editing::Animations::TipsForEditing::PreviewingBlends}
\begin{guide}{Previewing Blend Animations during Editing}
\item Test
\end{guide}

\clearpage

\subsection{GFS Properties}
\label{SECTION::GfsProperties}

\clearpage

\subsection{EPLs}
\label{SECTION::Edits::Epls}
EPL files are not currently editable in Blender. GFS files containing EPLs are loadable, but the EPL data will not be accessible. GFS files contained within EPLs can be extracted by other programs or by using this plugin as a library for the Python programming language, as described in section~\ref{SECTION::PythonLibrary}. These extracted GFS files may then be possible to open in Blender, edit, and then re-inject into the EPL files with other programs or \textit{via} the Python interface. Note however that, as of this version of the plugin, this is not well-tested nor intended to be an end-user feature. Users are welcome to use the library in this way, but are reminded that since it is an unsupported part of the library, pull requests that improve the feature are significantly more likely to get attention than issue tickets asking for improvements to it.

If sections of EPLs can be sensibly imported and exported from Blender, such as submodels, EPLs may be better supported in future versions of the plugin.

\clearpage

\section{BlenderToolsForGFS as a Python Library}
\label{SECTION::PythonLibrary}
\textbf{NOTE: This section is for users of the Python programming language who wish to write scripts using the plugin. A user will not get much out of this section without some knowledge of Python, which you are encouraged to learn using the wealth of information freely available on the internet if you need to use the library's scripting API.}\\\\
\noindent
The Blender plugin can also be used as a plain Python library for manipulating GFS and EPL files. This is possible because all Blender-related code is not loaded by default when initialising the plugin, and because the file-editing code is well-separated from the Blender interface. Furthermore, the non-Blender part of the plugin only utilises the standard library, and therefore no additional third-party libraries are required in order to use the plugin as a library. Using the plugin in this manner is outlined in section \ref{SECTION::UsingAsAPythonLibrary}.

If you have the `bpy' package installed, you can activate the plugin's Blender interface and write Blender scripts outside of Blender. You can also write these scripts inside Blender with the scripting feature. Notes on how to use the plugin in this manner are given in \ref{SECTION::UsingAsABlenderPlugin}.

\subsection{Python Library Usage}
\label{SECTION::UsingAsAPythonLibrary}

\subsection{Blender Script Usage}
\label{SECTION::UsingAsABlenderPlugin}

\clearpage

\end{document}
